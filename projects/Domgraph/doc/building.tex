\section{Building Utool} \label{sec:building}

So far, we have focused on how to use the complete Utool system as you
download it, unchanged. Even if you want to use the Domgraph library
that constitutes the main part of Utool from your own Java program,
you can simply add the Jar file to your classpath and use it. But
there may be a point at which you want to recompile Utool, e.g.\
because you make Utool aware of a new codec, or you have a better idea
for how to solve dominance graph. For such cases, we will now document
how to unpack the Utool source distribution and recompile it.



\subsection{The Utool source distribution}

The Jar file \verb?Utool-<version>.jar? which you probably downloaded
from the website only contains the compiled class files that are
necessary for running the programme. This is the standard distribution
because it is only 200k in size, and thus downloads and loads quickly.

If you want to recompile (parts of) Utool, you will need to download
the source distribution. The source distribution has a filename of the
form \verb?Utool-src-<version>.jar?, and is about 2 megabytes in
size. In addition to the Utool classes, it contains the source code of
Utool/Domgraph, all classes of the iText, JGraphT, JGraph, and Getopt
libraries that we use, and the javacc and testng Jars that are used in
compiling Utool. The files in the source distribution Jar are in
exactly the same locations as in the ordinary Jar, so in particular
you can use the command \verb?java -jar Utool-src-<version>.jar? to
run Utool.

In addition, you need the Java SDK 5.0 or higher, and you need Apache
Ant 1.6 or higher.

If you want to recompile the source distribution, the first step is to
unpack the Jar file in a new directory:

\begin{verbatim}
$ jar xf Utool-src-<version>.jar
\end{verbatim}
%$

This will write the contents of the Jar into the current
directory. The second step is then to prepare for compiling by moving
all files to the directories where the build script expects them:

\begin{verbatim}
$ ant -f projects/Domgraph/build.xml prepare
\end{verbatim}
%$

At this point, you can add new Java files or edit existing ones. The
build file expects that new Java files are put into the subdirectories
of the \verb?src? directory that correspond to their packages. You can
then recompile the entire system as follows:

\begin{verbatim}
$ ant -f projects/Domgraph/build.xml 
\end{verbatim}
%$

This will create the file \verb?Utool-full-<version>.jar? in the
directory \verb?build/lib?, which you can run using \verb?java -jar?
as before. If you have access to the Jakarta BCEL library (we use
version 5.1), you can add this library to your classpath and run

\begin{verbatim}
$ ant -f projects/Domgraph/build.xml utool-compact
\end{verbatim}
%$

to build the (much smaller) file \verb?build/lib/Utool-<version>.jar?
which constitutes the main Utool distribution.




\subsection{The package structure and the apidocs}

The Domgraph library (which we distribute under an LGPL licence)
consists of the following packages:

\begin{tabular}{l|l}
package & description \\\hline
\verb?de.saar.chorus.domgraph? & metadata (e.g.\ Utool version) \\
\verb?de.saar.chorus.domgraph.graph? & labelled dominance graphs \\
\verb?de.saar.chorus.domgraph.chart? & dominance charts and the solver
\\
\verb?de.saar.chorus.domgraph.codec? & the abstract base classes for
codecs \\
\verb?de.saar.chorus.domgraph.codec.*? & the individual codec classes
\\
\verb?de.saar.chorus.domgraph.equivalence? & redundancy elimination
algorithms
\end{tabular}

In addition, the main Utool program (which we distribute under a GPL
licence because it depends on Gnu Getopt) is defined in the package
\url{de.saar.chorus.domgraph.utool}. The classes for Ubench
\todo{licence?} are defined in \url{de.saar.chorus.ubench} and its
subpackages.

The apidocs of all classes can be found on the Utool website
\todo{url}. If anything about them is unclear, we would appreciate
your feedback to help improve the documentation.





%%% Local Variables: 
%%% mode: latex
%%% TeX-master: "0"
%%% TeX-command-default: "LaTeX"
%%% End: 
