%!TEX root = 0.tex

\section{Conclusion}  \label{sec:conclusion}

In this document, we have described Utool, the Swiss Army Knife of
Underspecification. In particular, we have explained how to use Utool
and how to extend Utool, and listed some tips and experiences on using
Utool in practice.

As this is the first version of this manual, we are sure that there
are many things that we can improve. We would therefore greatly
appreciate any comments or suggestions that you might have. Please
send them to \url{koller@coli.uni-sb.de}, and we will try to take them
into account for the next version. 

We believe that Utool 3.0 will do a good job at supporting researchers
who work on underspecification. Nevertheless, we will continue to
improve it, and have already started working on version 3.1. This new
version will mostly focus on improvements to the Utool architecture
that won't be immediately visible to the end user. However, we are
also planning to improve the functionality of the Ubench GUI and
implement new codecs. In addition, we would be thrilled to hear your
ideas for things that could be improved, as well as bug
reports. Please send these to \url{koller@coli.uni-sb.de} as well.

We hope you will enjoy using Utool!





%%% Local Variables: 
%%% mode: latex
%%% TeX-master: "0"
%%% TeX-command-default: "LaTeX"
%%% End: 
