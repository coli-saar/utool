\section{Using Utool}  \label{sec:operations}

In the current release, Utool supports six commands: \verb?solve?,
\verb?solvable?, \verb?classify?, \verb?convert?, \verb?server?,
\verb?display?. Each of these commands can be used from the
command-line or in the Utool Server (but running the \verb?server?
command in the Utool Server doesn't do anything). In addition, Utool
supports several auxiliary pseudo-commands, which display help
information. 

In this section, we describe these commands in some detail.


\subsection{Command-line interface}
When Utool is run from the command-line, it executes the single
command you specify on the command line and then exits. For instance,
the following call executes one \verb?solvable? command and outputs
some information about it:

\begin{verbatim}
$ utool solvable -s chain-3.clls
\end{verbatim}
% $

Commands typically require \emph{arguments} (in the example, the
filename \verb?chain-3.clls? of the USR that we want to solve), and
accept certain \emph{options} (here, the \verb?-s? option, which
instructs Utool to display statistical information). These arguments
and options are written after the command on the command line.


\subsection{Server mode} \label{sec:operations-server}

Alternatively, you can run Utool in server mode. In this mode, it will
not execute any commands at first. Instead, it will accept network
connections on a specific socket. Each time it is sent a command on
this socket, it will execute this command, send the results back over
the socket, and then close the socket. But the same Utool process
keeps running and can execute many commands during its lifetime. 

You start Utool in server mode by making Utool execute the
\verb?server? command:

\begin{verbatim}
$ utool server
\end{verbatim}
%$

The server doesn't output anything on the console on which you started
it -- unless you specified the \verb?--logging? command-line option
and no name for the logfile, in which case it will write some
information to standard error. 

By default, the server will listen for socket connections on port
2802, but the port can be specified using the \verb?--port?
option. The protocol for communicating with the server is as follows:


\begin{enumerate}
\item The client sends a command in an XML element of the following
form: 
\begin{verbatim}
<utool cmd="..." (more options)>
  (arguments of the command)
</utool>
\end{verbatim}

The command, options, and arguments in this XML element correspond to
the command, options, and arguments of a single run of the
command-line tool.

\item The client shuts down the half of the socket that is responsible
for sending data to the server (i.e.\ the ``output'' direction). Many
socket libraries (e.g.\ the standard libraries for C and Perl) offer a
\verb?shutdown? function for this purpose. Notice that \verb?shutdown?
is different from \verb?close? because it only closes one direction of
the socket; you can't close the socket yet because you still need to
receive the response from the server.

\item Once the client-to-server socket is shut down, the Utool Server
processes the command.

\item If the command is executed successfully, the server will respond
with a message of the following form:

\begin{verbatim}
<result .... />
\end{verbatim}

The particular attributes of this element depend on the command, and
are described below. If an error occurred, the response will be a
message of the following form instead:

\begin{verbatim}
<error code="..." explanation="..." />
\end{verbatim}

Here the ``code'' attribute will be a numeric error code, and the
``explanation'' attribute will be a plain-text explanation of the
error that occurred.

\item The server closes the socket.
\end{enumerate}


If you just use Utool for a single call or experimentation, the
command-line mode will typically be easier to use. However, if you
need to process many Utool commands from a programme, it can be
dramatically more efficient to keep a single Utool Server process
running and send the commands to the server (see
Section~\ref{sec:practice-some-practical-tips}). We present a demo
client for the Utool Server in Section~\ref{sec:practice-client}.


\subsection{Passing USRs}

Most commands require the user to specify an USR that should be
processed.

In command-line mode, you pass a specification of the USR as a
command-line argument. Most input codecs interpret this specification
string as the filename from which the USR should be read; in the
example above, we used the file name \verb?chain-3.clls?. However, it
is up to the input codec to decide how the argument string should be
interpreted; the \verb?chain? codec, for example, interprets the
string as the chain length and not as a filename.

In server mode, you can't pass a filename because the server may run
on a different machine than the client and may not have access to its
filesystem. Instead, you pass the USR directly as an attribute of a
\verb?usr? element that is embedded into the \verb?utool? element,
like so:

\begin{verbatim}
<utool cmd="solvable">
  <usr codec="domcon-oz" string="[label(x f(y)) ...]" />
</utool>
\end{verbatim}

Notice that the attribute values must be valid XML attribute
strings. This means that you must replace special characters by
their respective character entities (see also
Section~\ref{sec:practice} for tips on this).


\subsection{Exit codes}

Each execution of a Utool command returns an \emph{exit code}. The
command-line version of Utool will return this as the programme exit
code, which you can access e.g.\ in the \texttt{\$?} variable in a
Bash shell. The server version will return the exit code in the
\verb?code? attribute in the responses for many commands (and always
when it reports an error).

Exit codes are numbers between 0 and 255. They are split up into
ranges with different meanings as follows:

\begin{tabular}{l|l}
exit codes & meaning \\\hline
0 -- 127 & command was executed successfully \\
128 -- 191 & an error occurred in the main programme \\
192 -- 223 & an error occurred in the input codec \\
224 -- 255 & an error occurred in the output codec
\end{tabular}

The exit codes for successful termination of a command are documented
with the commands below. Among the codec error codes, the code 192 is
special because it always signifies a parsing error in the input
codec. The codes between 193 and 223 denote \emph{semantic errors} in
the input codec; they and all output codec error codes are documented
with the codecs in Section~\ref{sec:codecs}. The error codes for the
main programme are as follows:

\begin{tabular}{l|l}
exit code & meaning \\\hline
128 & file I/O error \\
129 & network I/O error \\
130 & Ubench encountered an error while laying out a graph \\

140 & error while configuring an (XML) parser \\
141 & the command you specified was not recognised \\
142 & error while registering a codec \\

150 & you didn't specify a USR, but the command requires one \\
151 & you didn't specify an input codec, and Utool cannot guess it \\
152 & there is no input codec of the name you specified \\
153 & the input graph is not weakly normal or not compactifiable \\

160 & you didn't specify an output codec, and Utool cannot guess it \\
161 & there is no output codec of the name you specified \\

170 & error while parsing an equivalence specification
\end{tabular}




\subsection{The commands supported by Utool}

We will now go through the seven main commands and describe what each
command does and what options it takes.


\subsubsection{Solvable}

This command converts the input USR into a dominance graph and checks
whether this graph is solvable, i.e.\ has any
solutions. Linguistically, this corresponds to checking whether the
sentence has any readings (ideally, it should!).

Solvability is determined by computing a dominance chart as described
in \cite{KolTha05b}. This is typically much more efficient than
actually \emph{solving} the graph, i.e.\ enumerating its solved forms,
because the chart is exponentially smaller than the set of all solved
forms. This command only makes a yes/no decision about solvability and
thus doesn't have to enumerate all the solved forms. If you want them,
see the ``solve'' command below.


\paragraph{Result.}
In command-line mode, the Utool process will terminate with an exit
code of 1 if the graph was solvable. It will terminate with an exit
code of 0 if it wasn't. 

In server mode, Utool will send a reply of the following form:
\begin{verbatim}
<result solvable='true' fragments='7' count='5' chartsize='10' time='30' />
\end{verbatim}

The \verb?solvable? attribute contains the string \verb?true? if the
graph was solvable, and \verb?false? otherwise. The other attributes
contain statistical information: the number of fragments of the graph,
the number of solved forms, the number of splits in the chart, and the
time in milliseconds it took to compute the chart.


\paragraph{Options.}
The \verb?solvable? command can take the following options:
\begin{itemize}
\item \textbf{input-codec:} You can specify an input codec for this
command. If you don't do this, the command-line version (but not the
server) will try to guess the appropriate input codec from the
filename extension if possible.

In command-line mode, specify the input codec with the option
\verb?--input-codec <codecname>? or \verb?-I <codecname>?. In server
mode, specify it as the \verb?codec? attribute of the \verb?usr?
element.

\item \textbf{display-statistics:} You can make Utool display more detailed
statistics when you call it on the command-line by passing the
\verb?--display-statistics? or \verb?-s? option. All such statistics
information will be written to standard error.
\end{itemize}




\subsubsection{Solve}

This command converts the input USR into a dominance graph, computes
the solved forms of this graph, and outputs them using the given
output codec. It computes a dominance chart, and if the graph was
solvable, proceeds to enumerate all solved forms.


\paragraph{Result.}
In command-line mode, Utool will output all solved forms. By default,
it will write them to standard output, but there is a command-line
option for redirecting them into a file. This command will always
terminate with an exit code of 0 if no errors occurred.

In server mode, Utool will send a reply of the following form:
\begin{verbatim}
<result solvable='true' fragments='7' count='5' chartsize='10'
        time-chart='30' time-extraction='100'>
  <solution string='....' />
  <solution string='....' />
</result>
\end{verbatim}

The attributes of the \verb?result? element are as for the
\verb?solvable? command, except that the runtime is now reported
separately for computing the chart and for enumerating (extracting)
the solved forms. The solutions are returned in attributes of
\verb?solution? elements below the \verb?result? element. Notice that
you may need to resolve XML character entities that were used in the
attribute value strings.

\paragraph{Options.}
The \verb?solve? command can take the following options:
\begin{itemize}
\item \textbf{input-codec:} see \verb?solvable?
\item \textbf{output-codec:} You can specify the output codec which
should be used to encode the solved forms. If you don't specify the
output codec explicitly, the command-line tool will first try to guess
the output codec from the output filename if you specify one. If this
doesn't work, it will try to use the output codec with the same name
as the input codec, if it exists. The server will not make such
guesses and expects you to specify the codec explicitly in any case.

In command-line mode, specify the output codec with the option
\verb?--output-codec <codecname>? or \verb?-O <codecname>?. In server
mode, specify it by passing the codec name as the \verb?output-codec?
attribute of the main \verb?utool? element.

\item \textbf{output:} By default, the command-line tool will write
the encoded solved forms to standard output. You can override this by
specifying an output file with the option \verb?--output <filename>?
or \verb?-o <filename>?. This option is not applicable in server mode
because it doesn't write into files anyway.

\item \textbf{no-output:} You can instruct Utool not to output the
actual solved forms by specifying the ``no-output'' option. Utool will
still \emph{compute} all solved forms in this case, it will simply not
\emph{output} them. This can be useful for runtime
measurements. Specify the option \verb?--no-output? or \verb?-n? on
the command line. In server mode, you get this effect if you simply
don't specify any output codec in the \verb?utool? element.

\item \textbf{display-statistics:} see \verb?solvable? 
\end{itemize}






\subsubsection{Convert}

This command reads a USR and outputs it again. The point about this
operation is that the input and output codecs may be different. This
means that you can use it to convert USRs from one underspecification
formalism to another (to the extent that this is supported by theory).

\paragraph{Result.}
In command-line mode, Utool will output the USR using the specified
output codec. By default, it will write it to standard output, but you
can again redirect the output to a file. The command always returns an
exit code 0 on successful completion.

In server mode, Utool will send a reply of the following form:
\begin{verbatim}
<result usr='....' />
\end{verbatim}

The string in the \verb?usr? attribute is the converted USR. Remember
that you may need to resolve XML character entities that were used in
the attribute value strings.

\paragraph{Options.}
The \verb?convert? command can take the following options:
\begin{itemize}
\item \textbf{input-codec:} see \verb?solvable? 
\item \textbf{output-codec:} see \verb?solve? 
\item \textbf{output:} see \verb?solve?
\item \textbf{no-output:} see \verb?solve?
\item \textbf{display-statistics:} see \verb?solvable?
\end{itemize}




\subsubsection{Classify} \label{sec:op-classify}

The \verb?classify? command checks whether a dominance graph belongs
to certain particularly well-behaved classes. It currently
distinguishes the following classes:

\begin{itemize}
\item \emph{weakly normal}: A dominance graph is weakly normal
\cite{bodirsky-weakly-normal-constraints} if the tree edges form a
forest and all dominance edges go into roots.
\item \emph{normal}: A weakly normal dominance graph is normal
\cite{Althaus-J.Algo.} if all dominance edges also start in
(unlabelled) holes.
\item \emph{compact}: A weakly normal dominance graph is compact if
all fragments have depth zero or one, i.e.\ no node has both incoming
and outgoing tree edges.
\item \emph{compactifiable:} Many weakly normal graphs that are not
compact can nevertheless be made compact by removing internal nodes
and labelled leaves of fragments, and adding tree edges from the roots
to the holes. In particular, all normal graphs are compactifiable. If
a graph is compactifiable, it is guaranteed that there is a one-to-one
correspondence between the solved forms of the compactified graph and
the solved forms of the original graph.
\item \emph{hypernormally connected:} A normal graph is hypernormally
connected \cite{KolNieTha03,Koller04} if any pair of nodes is
connected by a simple hypernormal path, where a hypernormal path is an
undirected path that uses no two dominance edges that are adjacent to
the same hole. This graph class is a bit abstract, but immensely
useful because these graphs (a) have a lot of nice structural
properties, and (b) we believe that all graphs that are currently used
in underspecification are or should be hypernormally connected
\cite{FucKolNieTha04}. Because of (a), the translations of MRS and
Hole Semantics into dominance graphs only work for USRs that translate
into hypernormally connected graphs.
\item \emph{leaf-labelled:} A weakly normal graph is leaf-labelled
\cite{KolNieTha03} if every (unlabelled) hole has an outgoing
dominance edge. If a graph is both hypernormally connected and
leaf-labelled, each of its solved forms has a configuration.
\end{itemize}


\paragraph{Result.}
The classes to which a graph belongs are encoded as the bit-wise OR of
the following values:

\begin{center}
\begin{tabular}{l|l}
class & bit value \\ \hline
weakly normal & 1 \\
normal & 2 \\
compact & 4 \\
compactifiable & 8 \\
hypernormally connected & 16 \\
leaf-labelled & 32
\end{tabular}
\end{center}

The command-line tool will return this value as its exit code upon
successful completion.

The Utool Server will return a message of the following form:
\begin{verbatim}
<result code='63' weaklynormal='true' normal='true'
        compact='true' compactifiable='true'
        hypernormallyconnected='true' leaflabelled='true' />
\end{verbatim}

Here \verb?code? is the exit code described above, and the other
attributes are either \verb?true? or \verb?false?.


\paragraph{Options.}
The \verb?classify? command can take the following options:
\begin{itemize}
\item \textbf{input-codec:} see \verb?solvable? 
\item \textbf{display-statistics:} see \verb?solvable?
\end{itemize}



\subsubsection{Server}

The \verb?server? command starts a new Utool Server. By default, this
server listens to new connections on port 2802, but you can specify a
different port using the \verb?--port? option. This command is ignored
if Utool is already running in server mode; it can only be run from
the command line.

\paragraph{Result.} This command doesn't terminate by itself; you have
to shut down the server process by hand.

\paragraph{Options.} The \verb?server? command can take the following
options:

\begin{itemize}
\item \textbf{port:} By default, the server will listen on a socket on port
2802. You can specify a different port using the option
\verb?--port <number>? (or \verb?-p <number>?), where \verb?<number>?
is the port number you want.

\item \textbf{logging:} Using this option, you can make the server log
information about incoming commands and its responses. If you specify
the option \verb?--logging? (or \verb?-l?) by itself, the log messages
will be written to standard error. Alternatively, you can log the
messages into a file by specifying the filename:
\verb?--logging <filename>? or \verb?-l <filename>?.
\end{itemize}




\subsubsection{Display}

The \verb?display? command instructs Utool to display the input USR in
the Underspecification Workbench (Ubench). If the Utool process has
opened a Ubench window before, it will display the USR in a new tab of
the same window; otherwise it will open a new Ubench window first.

Unlike the previous commands which accept USRs as input, the input USR
argument is \emph{optional} in the \verb?display? command. This means
that you may specify an input USR (in which case it is displayed right
away), or you can call \verb?display? without arguments. In this case,
an empty Ubench window is displayed; you can still open USRs from the
File/Open menu or by sending further \verb?display? commands to the
Utool Server.

Ubench offers a large subset of Utool's functionality from within a
convenient GUI. However, some functions are not accessible from Ubench
-- for example, Ubench doesn't support codec options yet, and you
can't use it to write all solved forms of a USR to a file in one
go. We plan to make this functionality available in Version 3.1.


\paragraph{Result.} This command doesn't terminate by itself; you have
to quit Ubench by choosing the File/Quit menu entry.

\paragraph{Options.} The \verb?display? command can take the following
options:

\begin{itemize}
\item \textbf{input-codec:} see \verb?solvable? 
\end{itemize}




\subsection{Pseudo-commands}

In addition to the six main commands, the command-line version of
Utool will accept a number of ``pseudo-commands'', which display help
information. If you call Utool with a pseudo-command, it executes the
pseudo-command and terminates immediately. You cannot specify both a
command and a pseudo-command at the same time.

The following pseudo-commands exist:

\begin{itemize}
\item \verb?help [command]?: If you specify a command, this will
display a brief help message for that command. Otherwise, it will
display an overview over the possible commands.
\item \verb?--version?: Displays the version of Utool.
\item \verb?--display-codecs? or \verb?-d?: Displays the installed
codecs.
\item \verb?--help-options?: Displays an overview over some frequently
used options.
\end{itemize}




\subsection{Advanced options}

\subsubsection{Redundancy Elimination} \label{sec:redund-elim}

One classical challenge when working with scope ambiguities is that
structurally different readings may be logically equivalent. This
problem of ``spurious ambiguity'' is illustrated by sentences like the
following:

\begin{examples}
\item \label{ex:spurious}
A researcher of some company saw a sample of a product.
\item
  \label{ex:rondane-1262} 
  For travellers going to Finnmark there is a bus service from Oslo to
  Alta through Sweden. (Rondane 1262)
\item 
  \label{ex:rondane-892}
  We quickly put up the tents in the lee of a small hillside and cook
  for the first time in the open. (Rondane 892)
\end{examples}

Example (\ref{ex:spurious}) is an ambiguous sentence with fourteen
different readings, but these readings are all logically equivalent
because existential quantifiers in predicate logic can be permuted
with each other without changing the interpretation. Examples
(\ref{ex:rondane-1262}) and (\ref{ex:rondane-892}) are sentences from
the Rondane Treebank. The English Resource Grammar analyses them as
having 3960 and 480 scope readings, respectively; but the first
sentence is intuitively unambiguous, and the second one has two
readings that differ in the relative scope of ``the lee'' and ``a
small hillside''. This surprisingly high number of readings comes from
the fact that the ERG analyses all kinds of noun phrases, including
proper names and pronouns, as scope bearing operators, and many of
these don't in fact take scope. But the basic problem of spurious
ambiguity has haunted semanticists ever since Montague's Quantifier
Raising analysis.

Utool is able to efficiently eliminate spurious ambiguities to a
limited degree. It implements the redundancy elimination algorithm
described by \citeN{koller06}, which assumes that (logical)
equivalence is approximated as equivalence with respect to a system of
term equations. For efficiency reasons, Utool when run as a main
programme won't eliminate splits from the chart. Instead, it won't
even add eliminable splits to the chart in the first place. However,
the distribution includes classes for running a redundancy elimination
on the complete chart (see the
\url{de.saar.chorus.domgraph.equivalence} package).

In order to make Utool eliminate equivalences in an invocation from
the commandline, you must pass it the
\verb?--equivalences <equivfile>? or \verb?-e <equivfile>?
option. Here \verb?<equivfile>? is the name of a file which contains a
specification of the equation system. In server mode, you pass an
element \verb?<eliminate equations="..." />? as a child of the
\verb?utool? element, where the value of the \verb?equations?
attribute is the specification of the equation system.

An equation system specification is an XML document of the following
form:

\begin{verbatim}
<?xml version="1.0" ?>
<equivalences style="(some name)">   
  .....
</equivalences>
\end{verbatim}

You may nest two kinds of elements below the root element:
\begin{itemize}
\item \verb?equivalencegroup?: This element specifies a set of
label-hole pairs that can all be permuted with each other. For
instance, if you want to specify that existential quantifier in
first-order logic can be permuted with each other, regardless of
whether they are in each other's scopes or restrictions, you can use
the following specification:
\begin{verbatim}
<equivalencegroup>
  <quantifier label="exists" hole="0" />
  <quantifier label="exists" hole="1" />
</equivalencegroup>
\end{verbatim}

By contrast, universal quantifiers only permute with each other if
they are plugged into each other's scope:
\begin{verbatim}
<equivalencegroup>
  <quantifier label="every" hole="1" />
</equivalencegroup>
\end{verbatim}

You may have as many \verb?equivalencegroup? elements as you
like, and you may have as many entries in each \verb?equivalencegroup?
as you like.

\item \verb?permutesWithEverything?: This element expresses that a
certain quantifier permutes with every other quantifier. For instance,
you can state that a proper name as analysed by the ERG permutes with
every other quantifier as follows:

\begin{verbatim}
<permutesWithEverything label="proper_q" hole="1" />
\end{verbatim}
\end{itemize}

The \verb?hole? attribute specifies the index \todo{check
implementation wrt compactification}, starting at 0 for the leftmost
hole. 

An example equivalence system, which is appropriate for the MRSs
generated by the ERG, and which we used for the evaluation in
\cite{koller06}, is part of the Utool distribution (in
\url{projects/Domgraph/examples/erg-equivalences.xml}). 




\subsection{Options overview}

We conclude this section with an overview over all options.

\newcommand{\optiondesc}[4]{\item #1 \\ (Server mode: #2) \\
(applies to: #3) \\ \strut\\ #4}

\begin{itemize}
\optiondesc
{\texttt{--input-codec <codecname>} or \texttt{-I <codecname>}}
{\texttt{codec} attribute of the \texttt{usr} elements}
{\texttt{solvable}, \texttt{solve}, \texttt{classify}, \texttt{convert}}
{Specify the input codec.}

\optiondesc{\texttt{--input-codec-options <options>}}
{\texttt{codec-options} attribute of the \texttt{usr} elements}
{\texttt{solvable}, \texttt{solve}, \texttt{classify}, \texttt{convert}}
{If the selected input codec accepts options, use this option to
specify them.}

\optiondesc{\texttt{--output-codec <codecname>} or \texttt{-O <codecname>}}
{\texttt{output-codec} attribute of the \texttt{utool} element}
{\texttt{solve}, \texttt{convert}}
{Specify the output codec.}

\optiondesc{\texttt{--output-codec-options <options>}}
{\texttt{output-codec-options} attribute of the \texttt{utool} element}
{\texttt{solve}, \texttt{convert}}
{If the selected output codec accepts options, use this option to
specify them.}

\optiondesc{\texttt{--no-output} or \texttt{-n}}
{assumes this option if no output codec is specified}
{\texttt{solve}, \texttt{convert}}
{Compute the output, but don't display it (useful for runtime
measurements). If you specify this option, you don't need to specify
the output codec (and if you do, it is ignored). }

\optiondesc{\texttt{--output <filename>} or \texttt{-o <filename>}}
{not applicable: the server doesn't write to an output file}
{\texttt{solve}, \texttt{convert}}
{Write the output to the specified file, rather than to standard
output. If you use this option and don't specify an output codec
explicitly, Utool will try to guess the appropriate output codec from
the filename extension.}




\optiondesc{\texttt{--display-statistics} or \texttt{-s}}
{not applicable: the server reports statistics information anyway}
{\texttt{solvable}, \texttt{solve}, \texttt{classify}, \texttt{convert}}
{Display statistics information while executing the command, such as
information about the graph classification, chart size, and
runtimes. All statistics information is written to standard error.}


\optiondesc{\texttt{--equivalences <equivfile>} or
\texttt{-e <equivfile>}}
{specify an element of the form \texttt{<eliminate equations="..." />}
as a child of the \texttt{utool} element}
{\texttt{solvable}, \texttt{solve}}
{Run a redundancy elimination algorithm on the chart (see  Section~\ref{sec:redund-elim}).} 
\end{itemize}




%%% Local Variables: 
%%% mode: latex
%%% TeX-master: "0"
%%% TeX-command-default: "LaTeX"
%%% End: 
