\section{Codecs}  \label{sec:codecs}

\todo{document exit codes for all codecs}

Utool is intended as a ``Swiss Army Knife'' for working with
underspecified representations, and one of our design goals was
therefore to make it compatible with as many current
underspecification formalisms as possible. This is not a trivial task,
as different formalisms typically differ in crucial details, and naive
translations from one formalism to another are typically incorrect for
pathological inputs.

As we have explained above, Utool internally works with labelled
dominance graphs and uses \emph{codecs} to translate between these
graphs and the actual USRs that the user sees. An \emph{input codec}
transforms a USR in a certain format into an equivalent labelled
dominance graph, and an \emph{output codec} transforms a labelled
dominance graph into a USR in a certain format. There are two
different types of output codecs: \emph{graph output codecs} are
capable of encoding arbitrary USRs, whereas \emph{plugging output
codecs} can only encode solved forms. \todo{das ist nicht genau die
richtige definition; die codectypen unterscheiden sich eigentlich nur
in den eingabedaten, und das ist fuer den end-user eigentlich total
egal}

Some codecs are quite simple; for instance, the \verb?domcon-oz? and
\verb?domcon-gxl? input and output codecs are simply deal with
syntaxes for labelled dominance graphs. However, there are also input
codecs for MRS and Hole Semantics that do quite a bit of work to
compute correct dominance graphs, and rely on nontrivial formal
results; and there are output codecs for reading the solutions as
terms or translating them into graph file formats that can be
displayed using external viewers.

In addition, the Codec API is quite simple, and writing your own codec
doesn't require a deep knowledge of the rest of the Utool system. As
long as you observe certain rules about, say, the labelled dominance
graph that your input codec computes for your own underspecification
formalism, the rest of the Utool system will simply cooperate with
your new codec.

Below, we will go through the input and output codecs in the current
Utool distribution one by one, explain the formal background and the
concrete syntax we assume, and discuss their limitations.



\subsection{The domcon-oz codecs}

The \verb?domcon-oz? codecs deal with string representations of
labelled dominance graphs as lists of terms of the Oz programming
language. An example USR that these codecs can deal with looks as
follows:

\begin{verbatim}
[label(x f(x1)) label(y g(y1)) label(z a) dom(x1 z) dom(y1 z)]
\end{verbatim}

As you can see, USRs are lists of terms whose head symbols are either
\verb?label? or \verb?dom?. The term \verb?label(x f(x1 ... xn))?
expresses that the node \verb?x? should have the label \verb?f? and
its children (over tree edges) should be the nodes \verb?x1? to
\verb?xn?, from left to right. The term \verb?dom(x y)? expresses that
there should be a dominance edge from the node \verb?x? to the node
\verb?y?. All labels and node names must be Oz atoms (essentially,
this means that they must start with a lowercase letter).

This format was first used in the old CHORUS demo (in 1999 or so),
which was written in Mozart Oz. At the time, it was used to represent
dominance \emph{constraints} \cite{clls2000}, but as we know today,
weakly normal dominance constraints can be seen as dominance graphs
quite directly \cite{Koller04}. If we want to read a \verb?domcon-oz?
USR as a weakly normal dominance constraints, we can read \verb?label?
terms as labelling atoms, \verb?dom? terms as dominance atoms, and
silently assume an inequality atom $x \neq y$ for any two symbols $x$
and $y$ that appear as the first argument of a \verb?label? term.

Both the input and the output codec require that the dominance graph
is weakly normal. If it isn't, they raise a
\verb?MalformedDomgraphException?.

The \verb?domcon-oz? input codec considers lines that start with a
percent symbol as comments and ignores them. The output codec is a
graph output codec which will encode arbitrary graphs. These codecs
are associated with the filename extension \verb?.clls?.



\subsection{The domcon-gxl codecs}

The \verb?domcon-gxl? codecs deal with representations of a labelled
dominance graph in the GXL graph representation language. GXL
(\url{http://www.gupro.de/GXL/}) is a XML-based standard language for
this purpose. A USR in this format looks as follows:

\begin{verbatim}
<gxl xmlns:xlink="http://www.w3.org/1999/xlink">
   <graph id="utool-graph" edgeids="true" hypergraph="false" edgemode="directed">
      <node id="x">
         <type xlink:href="root" />
         <attr name="label"><string>f</string></attr>
      </node>
      <edge from="x" to="x1" id="edge0">
        <type xlink:href="solid" />
      </edge>

      <node id="x1">
         <type xlink:href="hole" />
      </node>
      <edge from="x1" to="z" id="edge3">
        <type xlink:href="dominance" />
      </edge>

      <node id="z">
         <type xlink:href="leaf" />
         <attr name="label"><string>a</string></attr>
      </node>
   </graph>
</gxl>
\end{verbatim}

The USR specifies nodes (using \verb?node? elements), which can have
the type \verb?root? (a node with outgoing tree edges), \verb?hole? (a
node with incoming but no outgoing tree edges), or \verb?leaf? (a node
with not adjacent tree edges). Each node which is not a hole must have
an embedded \verb?attr? element which specifies its label. In
addition, the USR specifies edges, which can have the types
\verb?solid? or \verb?dominance?.

Both the input and the output codec require that the dominance graph
is weakly normal. If it isn't, they raise a
\verb?MalformedDomgraphException?.

The \verb?domcon-gxl? codecs are intended primarily as a portable
exchange format for labelled dominance graphs. The output codec is a
graph output codec which will encode arbitrary graphs. These codecs
are associated with the filename extension \verb?.dc.xml?.



\subsection{The mrs-prolog input codec}

\cite{mrs-dom}

\todo{Stefan}

\subsection{The mrs-xml input codec}

\todo{Stefan}



\subsection{The holesem-comsem input codec}

The \verb?holesem-comsem? input codec can read USRs of the Hole
Semantics formalism. Hole Semantics \cite{Bos96} is a rather popular
underspecification formalism because it is conceptually very
accessible (underspecify formulas by allowing them to have holes which
can be plugged by other formulas). We assume the concrete syntax used
in the Prolog system that accompanies the Computational Semantics
textbook \cite{blackburn05:_repres_infer_natur_languag}; that is, it
should be possible to use all USRs generated by the book software with
this codec. USRs in this syntax are Prolog terms that look e.g.\ as
follows:

\begin{verbatim}
some(A, some(B, some(C, some(X, and(label(A), and(hole(B),
and(label(C), and(some(A, X, B), and(pred1(C,man,X), leq(C,B))))))))))
\end{verbatim}

Here \verb?A? and \verb?C? are labels, \verb?B? is a hole, and
\verb?X? is a object variable of the intended semantic representation,
which will be bound by an existential quantifier. All four symbols are
introduced by the outer \verb?some? terms. In addition, the term
\verb?some(A,X,B)? expresses that \verb?A? is labelled with $\exists
X. \verb?B?$, and the term \verb?pred1(C,man,X)? expresses that
\verb?C? is labelled with $man(X)$. The term \verb?leq(C,B)? says that
the label \verb?C? must be below the hole \verb?B?.

This codec makes use of the theoretical result that Hole Semantics
USRs that are hypernormally connected and leaf-labelled can be
translated into equivalent labelled dominance graphs
\cite{KolNieTha03}. The translation itself is not that complicated,
but the correctness proof is not trivial. The codec checks whether the
resulting graph is normal, leaf-labelled, and hypernormally
connected. If it isn't, it throws a \verb?MalformedDomgraphException?
with one of the following error codes:

\begin{tabular}{lll}
code & symbolic name & meaning \\ \hline
193 &  \verb?ERROR_GRAPH_NOT_NORMAL? & resulting graph is not normal \\
194 &  \verb?ERROR_GRAPH_NOT_HNC? & resulting graph is not hypernormally
connected \\
195 &  \verb?ERROR_GRAPH_NOT_LEAF_LABELLED? & resulting graph is not
leaf-labelled \\
196 &  \verb?ERROR_MULTIPLE_PARENTS? & graph has a node with more than one
parent 
\end{tabular}

Although every normal, hypernormally connected, and leaf-labelled
dominance graph can in principle be translated back into an equivalent
Hole Semantics USR \cite{KolNieTha03}, there is no corresponding
output codec. This is because the concrete syntax imposes a number of
very inconvenient restrictions on the USRs that it can express. For
instance, all nodes that are not holes must have exactly one or two
children via tree edges, and all non-holes whose children are not
object-level variables must be labelled with a connective of predicate
logic. This has the consequence that only a tiny minority of all
labelled dominance graphs can actually be encoded given this syntax,
which made the encoding more difficult to implement and debug than we
felt it was worth.

The input codec is associated with the filename extension \verb?.hs.pl?.




\subsection{The domcon-udraw output codec}

\todo{Stefan}

\todo{discuss option}




\subsection{The plugging output codecs}

The \verb?plugging-oz? and \verb?plugging-lkb? output codecs are
plugging output codecs which can be used to encode solved forms. Both
codecs display the solved form as a list of pairs (hole, root) that
specify the dominance edges in the solved form (in Hole Semantics
terminology, this is a \emph{plugging}). They differ only in the
concrete syntax:

\begin{itemize}
\item \verb?plugging-oz?: The output is an Oz list of lists of terms
  that looks as follows:
\begin{verbatim}
[[plug(xr2 y2) plug(xl2 x1) plug(xl1 y0) plug(xr1 y1)]
 [plug(xl1 y0) plug(xr1 x2) plug(xr2 y2) plug(xl2 y1)]]
\end{verbatim}
Each list consists of terms of the form \verb?plug(x y)? which encode
the dominance edges (here: from \verb?x? to \verb?y?) in the solved
form.
\item \verb?plugging-lkb?: The output is a Lisp list of lists of lists
  that looks as follows:
\begin{verbatim}
( ( (r2 r2 2) (2 r2 2) (l2 l2 1) (1 l2 1) (l1 l1 0) (0 l1 0) (r1 r1 1) (1 r1 1))
  ( (l1 l1 0) (0 l1 0) (r1 r1 2) (2 r1 2) (r2 r2 2) (2 r2 2) (l2 l2 1) (1 l2 1)) )
\end{verbatim}

\todo{Stefan}

\end{itemize}

\verb?plugging-oz? is intended as a convenient output format if you
only want to see the dominance edges. On the other hand,
\verb?plugging-lkb? is practically relevant because it displays
pluggings in the format that the LKB workbench expects from its own
MRS solver. This means that Utool with the \verb?mrs-prolog? input
codec and the \verb?plugging-lkb? output codec can be used as a
drop-in replacement for the LKB's own MRS solver (see also
Section~\ref{sec:integration-lkb}).

Both codecs are plugging codecs; that is, they are only intended to be
run on dominance graphs in solved form. Both codecs will also accept
any other dominance graph and will then display its dominance edges,
but it is less meaningful than displaying the dominance edges in a
solved form. The \verb?plugging-oz? codec is associated with the
filename extension \verb?.plug.oz?, and the \verb?plugging-lkb? codec
is associated with the extension \verb?.lkbplug.lisp?.



\subsection{The term output codecs}

The \verb?term-oz? and \verb?term-prolog? output codecs are plugging
output codecs which can be used to encode labelled dominance graphs in
simple solved form, i.e.\ trees in which each hole has exactly one
outgoing dominance edge. They traverse these trees top-down and print
the ground term that corresponds to the tree structure. Their output
looks as follows:
\begin{itemize}
\item \verb?term-oz?: The codec computes a collection of Oz terms, one
  for each solved form, separated by newlines:
\begin{verbatim}
f2(f1(a0 a1) a2)
f1(a0 f2(a1 a2))
\end{verbatim}
\item \verb?term-prolog?: The codec computes a collection of Prolog
  terms, one for each solved form, separated by newlines:
\begin{verbatim}
f2(f1(a0,a1),a2)
f1(a0,f2(a1,a2))
\end{verbatim}
\end{itemize}

The only difference between the two codecs is that the Prolog codec
separates arguments of a term by commas, whereas the Oz codec
separates them with whitespace. These two codecs are intended as
human-readable representations of solved forms.

Both codecs assume that the solved form they encode is simple and
leaf-labelled, and will throw a \verb?MalformedDomgraphException? if
it isn't.

\verb?term-oz? is associated with the filename extension \verb?.t.oz?, and
\verb?term-prolog? with \verb?.t.pl?.





\subsection{The chain input codec}

The \verb?chain? input codec will generate the pure chain
\cite{Koller04} of a given length. A chain is a zig-zag graph
consisting of upper and lower fragments that are connected by
dominance edges; the pure chain of length 3 is shown in
Fig.~\ref{fig:chain3}. Chains appear frequently as parts of
linguistically motivated USRs, and are therefore a nice basis for
benchmarking (e.g.\ in \cite{bodirsky-weakly-normal-constraints}).

The ``USRs'' that can be processed by this codec are simply string
representations of numbers (such as the string \verb?3?). The codec
will then generate the pure chain of this length. When used with the
command-line version of Utool, \verb?chain? behaves differently than
all other codecs discussed so far in that it doesn't interpret its
argument as a filename, but again directly as the chain length. This
means that you can use a Utool call as follows:

\begin{verbatim}
$ java -jar Utool.jar convert -I chain 3 -O domcon-oz
\end{verbatim}
%$

The codec will throw a \verb?ParserException? if the length string
isn't a number, and a \verb?MalformedDomgraphException? if the number
isn't positive. Because \verb?chain? doesn't read its USRs from files,
it is not associated with any filename extension.






\subsection{Writing your own codecs}

Although Utool comes with a collection of codecs that cover most
existing popular underspecification formalisms, there are some
formalisms we don't support (yet), and we can expect that other
formalisms will be developed in the future. To this end, it may be
helpful for you to write your own codec.

A codec is a class that is derived from one of the abstract base
classes \verb?InputCodec?, \verb?GraphOutputCodec?, or
\verb?PluggingOutputCodec?, all of which are in the package
\verb?de.saar.chorus.domgraph.codec?. Your own codec can be in any
package you like. You never need to create objects of your codec
class; Utool creates objects of your class as needed.

Your codec class has the task of writing a representation of a
dominance graph to a Java \verb?Writer? (if it is an output codec) or
of reading such a representation from a Java \verb?Reader? and
initialising a labelled dominance graph (if it is an input codec). In
order to do this, you must implement certain abstract methods that
depend on the respective base class. This is documented in more detail
in the API documentation of the codec base classes on the website,
which is also distributed with the Utool package.

Your class must also have either a public constructor that takes no
arguments, or a public constructor that takes a single argument of
type \verb?String?. If it defines the second type of constructor,
Utool will pass the codec options string to your constructor if any
were specified on the command line or in the XML command. If no codec
options were specified, it will pass the value \verb?null?. Utool will
always call the \verb?String? constructor if it exists, and fall back
to the argumentless constructor if it doesn't.

In addition, your codec class is expected to implement public static
methods \verb?getName()? and \verb?getExtension()?, both with return
type \verb?String?. These methods should return the codec name and the
filename extension that you want to associate with your codec.

Finally, in order to inform Utool of your codec's existence, you must
\emph{register} the codec. In the current release, this task is a bit
of a nuisance, as you must add a call to \verb?registerCodec? for your
codec class in each of the following places of the Utool source code:
\begin{itemize}
\item method \verb?registerAllCodecs? in
  \url{de.saar.chorus.domgraph.utool.CommandLineParser};
\item method \verb?registerAllCodecs? in
  \url{de.saar.chorus.domgraph.utool.XmlParser};
\item method \verb?registerAllCodecs? in
  \url{de.saar.chorus.ubench.gui.Main}. 
\end{itemize}

You must then recompile Utool as described in \todo{where?}. Make sure
that your compiled codec class is in your classpath both when you
compile and when you run Utool. Then run \verb?java -jar Utool.jar -d?
to check whether your codec was loaded by Utool.








%%% Local Variables: 
%%% mode: latex
%%% TeX-master: "0"
%%% TeX-command-default: "LaTeX"
%%% End: 
