
\section{A tutorial walkthrough}
\label{sec:tutorial}

Welcome to the Utool tutorial! In this tutorial, we will walk you
through some of the basic operations that Utool supports.



\subsection{Installation}

Utool is distributed as a Java package. This means that you can simply
download it from our website and run it as follows.

\begin{verbatim}
$ java -jar Utool.jar
Usage: java -jar Utool.jar <subcommand> [options] [args]
Type `utool help <subcommand>' for help on a specific subcommand.
Type `utool --help-options' for a list of global options.
Type `utool --display-codecs' for a list of supported codecs.

Available subcommands:
    solve        Solve an underspecified description.
    solvable     Check solvability without enumerating solutions.
    convert      Convert underspecified description from one format to another.
    classify     Check whether a description belongs to special classes.
    server       Start Utool in server mode.
    help         Display help on a command.

Utool/Java is the Swiss Army Knife of Underspecification (Java version).
For more information, see http://www.coli.uni-sb.de/projects/chorus/utool/
\end{verbatim}
%$

(We write \verb?$ command? %$
for commands that you type on a shell; everything else is the output
of the system.)

This assumes that you have installed Java 5.0 and it is in your path.

You do not need to unpack the Jar file to run it. For the purposes of
this tutorial, however, start by running the following command:

\begin{verbatim}
$ jar xf Utool.jar
\end{verbatim}
%$

This will unpack the contents of the Jar file. Most importantly, it
will create the directory \verb?projects/Domgraph/examples?, which
contains the example files we will use below.




\subsection{Checking whether an USR is solvable}

First of all, let's use Utool to determine for a given dominance graph
whether it is solvable. ``Solvable'' means that it is possible to
configure the fragments of the graph (i.e.\ those subgraphs which are
connected by solid ``tree'' edges) into a forest while realising the
(dotted) dominance edges as reachability in the tree.

\begin{figure}
\todo{draw me}
\todo{graph; solved form; configuration}
\caption{\texttt{chain3.clls}: The chain of length 3.
\label{fig:chain3}}
\end{figure}

We will do this for the graph specified in the file
\url{projects/Domgraph/examples/chain-3.clls} (shown in
Fig.~\ref{fig:chain3}). This graph is in fact solvable, and has five
solved forms.

The Utool command for checking solvability is called
\verb?solvable?. Thus you can check the example graph for solvability
as follows:

\begin{verbatim}
$ java -jar Utool.jar solvable -s projects/Domgraph/examples/chain-3.clls
The input graph is normal.
The input graph is compact.

Solving graph ... it is solvable.
Splits in chart: 10
Time to build chart: 60 ms
Number of solved forms: 5
\end{verbatim}
%$

The output states that the graph is solvable and has five solved
forms. It also contains some information about the input graph (it is
normal and compact), the chart data structure computed by the solver
(it contains ten splits), and the time it took to compute the
chart. In addition, the program terminated with an exit code of 1 to
signal that the graph was indeed solvable. If you ran Utool from a
bash shell, you can display this exit code with the command
``\verb!echo $?!''. %$

As you can see, the Utool command line consists of four parts:
\begin{itemize}
\item \verb?java -jar Utool.jar?: This instructs Java to load the Jar
file and run its main class. You can pass further arguments to the
Java VM by putting them before the \verb?-jar? option.
\item \verb?solvable?: Most of the time, a call to Utool will contain
exactly one \emph{command}. There are six commands -- \verb?solvable?,
\verb?solve?, \verb?convert?, \verb?classify?, \verb?server?, and
\verb?help? -- which perform different tasks. We will walk you through
all four commands here, and then describe them in more detail in
Section~\ref{sec:operations}.
\item \verb?-s?: After the command, you can specify \emph{options}. In
this case, we selected one option (\verb?-s? or
\verb?--display-statistics?), which was responsible for printing all
the output from the example run above. We could have left the option
away; then the Utool run would have done exactly the same, and
returned the same exit code, but it wouldn't have printed these
informational messages.
\item \verb?projects/Domgraph/examples/chain-3.clls?: This is the name
of the file from which the dominance graph should be read. The file
can contain a direct specification of a dominance graph (as in this
case), or it can contain an USR of some other underspecification
formalism which is then translated to a dominance graph
automatically.
\end{itemize}



\subsection{Enumerating solutions}

Now that we know that the dominance graph is solvable, let's have
Utool show us the five solutions. We do this using the \verb?solve?
command, like so:

\begin{verbatim}
$ java -jar Utool.jar solve -O term-prolog projects/Domgraph/examples/chain3.clls
f1(a0,f2(a1,f3(a2,a3)))
f1(a0,f3(f2(a1,a2),a3))
f2(f1(a0,a1),f3(a2,a3))
f3(f2(f1(a0,a1),a2),a3)
f3(f1(a0,f2(a1,a2)),a3)
\end{verbatim}
%$

The five solved forms of the graph are displayed as terms in Prolog
syntax: Utool transforms each solved form into a \emph{configuration}
by identifying roots and holes as shown in Fig.~\ref{fig:chain3}, and
then prints the resulting trees as ground terms. You should convince
yourself at this point that the graph indeed has exactly five solved
forms, and that the five terms shown above are such term
representations of the configurations.

We had to use a new command-line option in the \verb?solve? command:
\verb?-O term-prolog? (which is shorthand for
\verb?--output-codec term-prolog?). Once Utool has computed a solved
form, it needs to know how it should translate this solved form into a
string representation. This translation is handled by an \emph{output
codec}. A number of output codecs are distributed with Utool; you can
also implement your own output codec if you like. In the example, we
used the \verb?term-prolog? output codec, which maps solved forms into
configurations and these configurations into Prolog terms as explained
above.

Speaking of codecs: There are also \emph{input codecs}, which map from
a string representation of an USR into a dominance graph. In the
example, Utool read the string representation from the file with the
specified name (\url{chain3.clls}) and passed it to an input codec to
obtain a dominance graph from it. In this case, we didn't have to
specify the input codec explicity, because Utool knows that the
filename extension \verb?.clls? is associated with the
\verb?domcon-oz? input codec, and used this codec automatically. We
could also have specified the input codec explicitly using the
\verb?-I domcon-oz? or \verb?--input-codec domcon-oz? option. Codecs
can be much more powerful than the ones mentioned here, and are
described in depth in Section~\ref{sec:codecs}.

 
\subsection{Converting and classifying USRs}

Utool works with labelled dominance graphs internally; input codecs
are used to translate from other formalisms into labelled dominance
graphs, and output codecs are used to translate from labelled
dominance graphs into other formalisms. These translations can be
non-trivial, and we have proved them correct in several research
papers \cite{KolNieTha03,mrs-dom}.

But Utool allows you to convert USRs from one formalism into
another. This is conceptually quite simple: Read the USR in one
formalism, translating it into a labelled dominance graph, and then
output exactly the same USR using a different output codec.

Let's look at an example again. The file
\url{projects/Domgraph/examples/rondane-1.mrs.pl} is an USR in the
Minimal Recursion Semantics formalism \cite{CopFliSag97}. This MRS is
computed by the English Resource Grammar
\cite{Copestake&Flickinger:LKB} for the first sentence in the Rondane
Treebank, which is distributed with the ERG. Its filename extension is
\verb?.mrs.pl?, which indicates an MRS 






%%% Local Variables: 
%%% mode: latex
%%% TeX-master: "0"
%%% TeX-command-default: "LaTeX"
%%% End: 
